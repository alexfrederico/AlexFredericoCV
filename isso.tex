%%%%%%%%%%%%%%%%%
% This is an example CV created using altacv.cls (v1.1.5, 1 December 2018) written by
% LianTze Lim (liantze@gmail.com), based on the
% Cv created by BusinessInsider at http://www.businessinsider.my/a-sample-resume-for-marissa-mayer-2016-7/?r=US&IR=T
%
%% It may be distributed and/or modified under the
%% conditions of the LaTeX Project Public License, either version 1.3
%% of this license or (at your option) any later version.
%% The latest version of this license is in
%%    http://www.latex-project.org/lppl.txt
%% and version 1.3 or later is part of all distributions of LaTeX
%% version 2003/12/01 or later.
%%%%%%%%%%%%%%%%

%% If you are using \orcid or academicons
%% icons, make sure you have the academicons
%% option here, and compile with XeLaTeX
%% or LuaLaTeX.
% \documentclass[10pt,a4paper,academicons]{altacv}

%% Use the "normalphoto" option if you want a normal photo instead of cropped to a circle
% \documentclass[10pt,a4paper,normalphoto]{altacv}

\documentclass[10pt,a4paper,ragged2e]{altacv}

%% AltaCV uses the fontawesome and academicon fonts
%% and packages.
%% See texdoc.net/pkg/fontawecome and http://texdoc.net/pkg/academicons for full list of symbols. You MUST compile with XeLaTeX or LuaLaTeX if you want to use academicons.

% Change the page layout if you need to
\geometry{left=1cm,right=9cm,marginparwidth=6.8cm,marginparsep=1.2cm,top=1.25cm,bottom=1.25cm}

% Change the font if you want to, depending on whether
% you're using pdflatex or xelatex/lualatex
\ifxetexorluatex
  % If using xelatex or lualatex:
  \setmainfont{Carlito}
\else
  % If using pdflatex:
  \usepackage[utf8]{inputenc}
  \usepackage[T1]{fontenc}
  \usepackage[default]{lato}
\fi

\usepackage{hyperref}

% Change the colours if you want to
\definecolor{VividPurple}{HTML}{378ee6}%{3E0097}
\definecolor{SlateGrey}{HTML}{2E2E2E}
\definecolor{LightGrey}{HTML}{37474F}
\colorlet{heading}{VividPurple}
\colorlet{accent}{VividPurple}
\colorlet{emphasis}{SlateGrey}
\colorlet{body}{LightGrey}

% Change the bullets for itemize and rating marker
% for \cvskill if you want to
\renewcommand{\itemmarker}{{\small\textbullet}}
\renewcommand{\ratingmarker}{\faCircle}

\begin{document}
\name{Alex Frederico}
\tagline{Cientista da Computação}
% Cropped to square from https://en.wikipedia.org/wiki/Marissa_Mayer#/media/File:Marissa_Mayer_May_2014_(cropped).jpg, CC-BY 2.0
\photo{2.5cm}{profile-picture}
\personalinfo{%
  % Not all of these are required!
  % You can add your own with \printinfo{symbol}{detail}
    \email{alexfredericomfm@gmail.com}
    \phone{(85) 9.9705-8288}
    \location{Maracanaú, Ceará}
    \linkedin{
        \href{https://www.linkedin.com/in/alex-frederico-mathias-felix-de-melo-6682b5107/}{linkedin.com/in/Alex-Frederico-Mathias}}
    \github{
        \href{https://github.com/alexfrederico}{github.com/alexfrederico}}
%   \orcid{orcid.org/0000-0000-0000-0000} % Obviously making this up too. If you want to use this field (and also other academicons symbols), add "academicons" option to \documentclass{altacv}
}

%% Make the header extend all the way to the right, if you want.
\begin{fullwidth}
\makecvheader
\end{fullwidth}

%% Depending on your tastes, you may want to make fonts of itemize environments slightly smaller
\AtBeginEnvironment{itemize}{\small}

%% Provide the file name containing the sidebar contents as an optional parameter to \cvsection.
%% You can always just use \marginpar{...} if you do
%% not need to align the top of the contents to any
%% \cvsection title in the "main" bar.

%%%%%%%%%%%%%%%%%%%%%%%%%%%%%%%%%%%%%%%%%%%%%%%%%%%%%%%%%%%%%%%%%

\cvsection[page1sidebar]{Experiência}

\cvevent{Consultor de Automatização e Digitalização}{Companhia Industrial de Cimento Apodi \hfill Tempo integral}{ago. de 2020 -- o momento}{Fortaleza, Ceará}
\begin{comment}
\begin{itemize}
    \item Otimizar e manutenir inteligencia artificial que controla o moinho de cimento;
    \item Apresentar informações acionáveis para ajudar usuários finais corporativos a tomar decisões de negócios bem informadas.
\end{itemize}
\cvtag{Python} 
\cvtag{PostgreSQL}
\cvtag{ETL}
\cvtag{Qlik Sense}
\cvtag{Data visualization}
\cvtag{Mineração de dados}
\cvtag{Business Intelligence}
\end{comment}

\vspace{10px}

\cvevent{Técnico de Automatização e Integração}{Companhia Industrial de Cimento Apodi \hfill Tempo integral}{dez. de 2019 -- jul. de 2020}{Fortaleza, Ceará}
\begin{comment}
\begin{itemize}
    \item Otimizar e manutenir inteligencia artificial que controla o moinho de cimento;
    \item Apresentar informações acionáveis para ajudar usuários finais corporativos a tomar decisões de negócios bem informadas.
\end{itemize}
\cvtag{Python} 
\cvtag{PostgreSQL}
\cvtag{ETL}
\cvtag{Qlik Sense}
\cvtag{Data visualization}
\cvtag{Mineração de dados}
\cvtag{Business Intelligence}
\end{comment}

\vspace{10px}

\cvevent{Pesquisador e Elaborador de Software}{Companhia Industrial de Cimento Apodi \hfill Autônomo}{ago. de 2019 -- nov. de 2019}{Fortaleza, Ceará}
\begin{itemize}
    \item Otimizar e manutenir inteligencia artificial que controla o moinho de cimento;
    \item Apresentar informações acionáveis para ajudar usuários finais corporativos a tomar decisões de negócios bem informadas;
    \item Desenvolver dashboards que mostram métricas e indicadores importantes para alcançar objetivos e metas traçadas;
    \item Desenvolver extração de dados de diversos sistemas, transformação desses dados conforme regras de negócios e por fim o carregamento em um Data Warehouse;
\end{itemize}
\cvtag{Python} 
\cvtag{PostgreSQL}
\cvtag{ETL}
\cvtag{Qlik Sense}
\cvtag{Data visualization}
\cvtag{Mineração de dados}
\cvtag{Business Intelligence}

\vspace{10px}

\cvevent{Cientista de Dados}{Companhia Industrial de Cimento Apodi \hfill Estágio}{jun. de 2018 – jun. de 2019}{Pecém, Ceará}
\begin{itemize}
    \item Fazer a coleta e armazenamento dos dados de sensores de forma a gerar dados que podem ser manipulados para analises;
    \item Desenvolver dashboard para analises dos dados coletados usando medidas estatísticas para encontrar correlações, similaridade e anomalias;
    \item Utilizar algoritmos de detecção de anomalias e machine learning para a limpeza dos dados brutos coletados;
    \item Participar da equipe de cientistas de dados para automatizar um moinho de cimento.
\end{itemize}
\cvtag{Python} 
\cvtag{MongoDB}
\cvtag{Sklearn}
%\cvtag{Pepiline}
\cvtag{Seaborn}
\cvtag{Estatística}
%\cvtag{Linux}
\cvtag{Pandas}
\cvtag{Matplotlib}
\cvtag{Data visualization}
%\cvtag{TensorFlow}
\cvtag{Anomaly detection}
%\cvtag{Flask}
\cvtag{Git}

\vspace{10px}


\cvevent{Monitor de Inteligência Artificial}{Universidade Federal do Ceará \hfill Monitoria}{jan. de 2018 -- dez. de 2018}{Russas, Ceará}
\begin{itemize}
    \item Auxiliar o professor-orientador na realização de trabalhos práticos e experimentais, na preparação de material didático e em atividades de classe e/ou laboratório.
\end{itemize}
\cvtag{SMA}
\cvtag{Pathfinding}
\cvtag{Planejamento Automatizado}
\cvtag{Machine Learning}
\cvtag{NLP}
\cvtag{Sistemas Especialistas} 
\cvtag{Lógica Proposicional}

\vspace{10px}

\cvevent{Monitor de Programação Orientada a Objetos}{Universidade Federal do Ceará \hfill Monitoria}{jul. de 2016 -- dez. de 2017}{Russas, Ceará}
\begin{itemize}
    \item Participar das tarefas didáticas, inclusive na programação de aulas e em trabalhos escolares;
    \item Contribuir, juntamente com o professor-orientador, para a avaliação do andamento da disciplina ou da área.
\end{itemize}
\cvtag{Java} 
\cvtag{Teste Unitário}
\cvtag{POO}
\cvtag{UML}
\cvtag{Ensino Hibrido}

\vspace{10px}

%%%%%%%%%%%%%%%%%%%%%%%%%%%%%%%%%%%%%%%%%%%%%%%%%%%%%%%%%%%%%%%%%

\cvsection{Projetos Paralelos}

\cvevent{Twinstabot}{Um bot do Instagram que posta uma nuvem de palavras com os assuntos mais comentados do Twitter}{}{}
\begin{itemize}
    \item Desenvolver mineração dos dados do Twitter;
    \item Desenvolver algoritmo para o processamento de texto;
    \item Desenvolver algoritmo para classificação dos tweets;
\end{itemize}
\cvtag{Python} 
\cvtag{NLP}
\cvtag{Web Scraping}
\cvtag{Selenium}
\cvtag{Spacy}
\cvtag{NLKT}
\cvtag{Expressão regular}
\cvtag{Sklearn}
\cvtag{Git}

\vspace{10px}

%%%%%%%%%%%%%%%%%%%%%%%%%%%%%%%%%%%%%%%%%%%%%%%%%%%%%%%%%%%%%%%%%

\cvsection{Educação}

\cvevent{Bacharelado em Ciência da Computação}{Universidade Federal do Ceará}{jul. de 2015 -- jul. de 2019}{Russas, Ceará}
\begin{itemize}
    \item Trabalho de Conclusão de Curso: Avaliação da remoção de anomalias em dados coletados por sensores.
\end{itemize}
%\faMortarBoard Score: 13.53/20

\vspace{10px}
\cvevent{Curso Qlik Sense}{Cubotimize Solucoes em Inteligencia Tecnologica}{set. de 2019 -- set. de 2019}{Fortaleza, Ceará}
\begin{itemize}
    \item Análises - N1;
    \item Arquitetura - N2;
    \item Análises Avançadas - N3;
    \item Arquitetura Avançada - N4.
\end{itemize}

\clearpage

\end{document}
